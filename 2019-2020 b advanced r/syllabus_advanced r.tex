\documentclass[a4paper]{article}
\usepackage{fontspec}      %% подготавливает загрузку шрифтов Open Type, True Type и др.
\defaultfontfeatures{Ligatures={TeX},Renderer=Basic}  %% свойства шрифтов по умолчанию
\setmainfont[Ligatures={TeX,Historic},
	SmallCapsFont={Brill},
	SmallCapsFeatures={Letters=SmallCaps}]{Brill}
\usepackage{alltt}
\usepackage{geometry} % Простой способ задавать поля
	\geometry{top=25mm}
	\geometry{bottom=20mm}
	\geometry{left=25mm}
	\geometry{right=20mm}
	\geometry{headsep=25mm}
\usepackage{setspace} % Интерлиньяж
\onehalfspacing % Интерлиньяж 1.5
%\doublespacing % Интерлиньяж 2
%\singlespacing % Интерлиньяж 1
%\usepackage{natbib}
%\bibpunct[: ]{[}{]}{;}{a}{}{,}
\usepackage{enumitem}
\usepackage{array,tabularx,tabulary,booktabs} % Дополнительная работа с таблицами
\usepackage{longtable}  % Длинные таблицы
\usepackage{multirow} % Слияние строк в таблице
\usepackage{multicol} % Несколько колонок

\setlist{nosep}
\usepackage{titlesec}
\titlespacing*{\section}
{0pt}{2ex plus 0ex minus .2ex}{0ex plus .2ex}
\titlespacing*{\subsection}
{0pt}{2ex plus 0ex minus .2ex}{0ex plus .2ex}
\titlespacing*{\subsubsection}
{0pt}{2ex plus 0ex minus .2ex}{0ex plus .2ex}
\usepackage{sectsty}
\sectionfont{\centering}
\renewcommand{\thesection}{\Roman{section}.}
\renewcommand \thesubsection{\arabic{section}.\arabic{subsection}}

\begin{document}
\begin{center}
{\Large Программа дисциплины <<Анализ данных для лингвистов>>}
\end{center}
\begin{flushright}
\begin{tabular}{l}
Утверждена                          \\
Академическим советом ОП            \\
Протокол № 15 от  «28» июня 2018 г.
\end{tabular}
\end{flushright}
\begin{center}
\begin{tabular}{|l|l|}
\hline
Автор                         & Г. А. Мороз                    \\ \hline
Число кредитов                & 3                              \\ \hline
Контактная работа (час.)      & 36                             \\ \hline
Самостоятельная работа (час.) & 78                             \\ \hline
Курс                          & 3, 4                            \\ \hline
Формат изучения дисциплины    & без использования онлайн-курса \\ \hline
\end{tabular}
\end{center}
\section{ЦЕЛЬ И РЕЗУЛЬТАТЫ ОСВОЕНИЯ ДИСЦИПЛИНЫ}
Задачей курса <<Анализ данных для лингвистов>> является продолжение знакомства с различными методами анализа данных. Курс разделен на несколько тематических блоков: первый связан с применением байесовских статистических методов (байесовский апдейт, байесовский доверительный интервал, байесовский фактор, байесовкская эмпирическая оценка), второй связан с методами уменьшения размерности (PCA, LDA, CA, MCA), третий блок связан с методами кластеризации (k-means, иерархическая кластеризация, смешанные модели) и последний блок будет посвящен проблемам применения регрессионного анализа (регрессия со смешанными эффектами, обобщённая аддитивная модель).
\section{СОДЕРЖАНИЕ УЧЕБНОЙ ДИСЦИПЛИНЫ}
\begin{enumerate}
\item Работа с распределениями
\item Байесовские статистические методы
\item Кластеризация и смешанные модели
\item Уменьшение размерности
\item Проблемы применения регрессионного анализа
\end{enumerate}
\section{ОЦЕНИВАНИЕ}
Результирующая оценка выставляется по следующей формуле:
$$O_\textnormal{итоговая} = 14 \times \int^{\frac{1}{10}\times O_\textnormal{экзамен}}_{-\frac{1}{10}\times O_\textnormal{накопленная}} \frac{\exp(\frac{1}{2}(x-1)^2)}{\sqrt{2\pi}}dx + O_\textnormal{дополнительный балл}$$
Оценка за курс складывается из оценок за домашние работы ($O_\textnormal{накопленная}$), экзамен ($O_\textnormal{экзамен}$), а также дополнительный балл ($O_\textnormal{дополнительный балл}$), который присуждается студенту, первым указавшим на ошибку в формуле оценивания. Способ округления всех оценок: арифметический.
\section{ПРИМЕРЫ ОЦЕНОЧНЫХ СРЕДСТВ}
\begin{itemize}
\item Посчитайте значение правдоподобия распределения $\mathcal{N}(\mu = 22,\, \sigma^{2}=6)$ для двух наблюдений 57 и 43.
\item Проведите байесовский апдейт данных $Beta(11, 34)$, используя априорное распределение $Beta(23, 45)$, и посчитайте симметричный 95\% байесовский доверильный интервал и 95\% интервал максимальной плотности.
\end{itemize}
\section{РЕСУРСЫ}
\subsection{Основная литература}
\begin{itemize}
\item Albert, J. Bayesian computation with R / J. Albert. – 2nd ed. – Heidelberg; Dordrecht; London; New York: Springer, 2009. – 298 с. – (Use R!) . ISBN 978-0-387-92297-3.
\item  Fox, J. An R companion to applied regression / J. Fox, S. Weisberg. – 2nd ed. – Los Angeles [etc.]: SAGE Publications, 2011. – 449 с. ISBN 978-1-412-97514-8. 
\end{itemize} 
\subsection{Дополнительная литература}
\begin{itemize}
\item Greenacre, M. Correspondence analysis in practice / M. Greenacre. – 2nd ed. – Boca Raton; London; New York: Chapman \& Hall/CRC, 2007. – 280 с. – (Interdisciplinary statistics series) - ISBN 978-1-584-88616-7. 
\item  Doing Bayesian data analysis: a tutorial with R, JAGS, and Stan / J. K. Kruschke. – 2nd ed. – Amsterdam [etc.]: Elsevier, 2015. – 759 с. ISBN 978-0-12-405888-0. 
\item  Gries, S. T. Statistics for linguistics with R: a practical introduction / S. T. Gries. – 2nd rev. ed. – Berlin; Boston: De Gruyter Mouton, 2013. – 359 с. ISBN 978-3-11-030728-3.
\item  Gries S. T. Ten lectures on quantitative approaches in cognitive linguistics: corpus-linguistic, experimental, and statistical applications / S. T. Gries. – Leiden; Boston: Brill, 2017. – 211 c. – (Distinguished lectures in cognitive linguistics) . ISBN 9789004336216. 
\end{itemize}
\subsection{Программное обеспечение}
\begin{tabular}{|l|l|l|}
\hline
№ п/п &	Наименование	& Условия доступа \\ \hline
1	& R &	Свободно распространяемое ПО \\ \hline
2	& Rstudio &	Свободно распространяемое ПО \\ \hline
\end{tabular}
\subsection{Профессиональные базы данных, информационные справочные системы, интернет-ресурсы (электронные образовательные ресурсы)}

\begin{tabular}{|l|l|l|}
\hline
№ п/п &	Наименование	& Условия доступа \\ \hline
1	& Страница курса &	https://agricolamz.github.io/2019\_data\_analysis\_for\_linguists/ \\ \hline
2	& Introduction to attractor landscapes &	https://ncase.me/attractors/\\ \hline
\end{tabular}

\subsection{Материально-техническое обеспечение дисциплины}
Учебные аудитории для лекционных занятий по дисциплине обеспечивают использование и демонстрацию тематических иллюстраций, соответствующих программе дисциплины в составе:
\begin{itemize}
\item ПЭВМ с доступом в Интернет (операционная система, офисные программы, антивирусные программы);
\item мультимедийный проектор с дистанционным управлением.
\end{itemize}
Учебные аудитории для лабораторных и самостоятельных занятий по дисциплине оснащены ­­­­­­­­­­­­­­­­­­­­­­­­ ПЭВМ, с возможностью подключения к сети Интернет и доступом к электронной информационно-образовательной среде НИУ ВШЭ.  
\end{document}
