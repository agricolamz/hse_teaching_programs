\documentclass[a4paper]{article}
\usepackage{fontspec}      %% подготавливает загрузку шрифтов Open Type, True Type и др.
\defaultfontfeatures{Ligatures={TeX},Renderer=Basic}  %% свойства шрифтов по умолчанию
\setmainfont[Ligatures={TeX,Historic},
	SmallCapsFont={Brill},
	SmallCapsFeatures={Letters=SmallCaps}]{Brill}
\usepackage{alltt}
\usepackage{geometry} % Простой способ задавать поля
	\geometry{top=25mm}
	\geometry{bottom=20mm}
	\geometry{left=25mm}
	\geometry{right=20mm}
	\geometry{headsep=25mm}
\usepackage{setspace} % Интерлиньяж
\onehalfspacing % Интерлиньяж 1.5
%\doublespacing % Интерлиньяж 2
%\singlespacing % Интерлиньяж 1
%\usepackage{natbib}
%\bibpunct[: ]{[}{]}{;}{a}{}{,}
\usepackage{enumitem}
\usepackage{array,tabularx,tabulary,booktabs} % Дополнительная работа с таблицами
\usepackage{longtable}  % Длинные таблицы
\usepackage{multirow} % Слияние строк в таблице
\usepackage{multicol} % Несколько колонок

\setlist{nosep}
\usepackage{titlesec}
\titlespacing*{\section}
{0pt}{2ex plus 0ex minus .2ex}{0ex plus .2ex}
\titlespacing*{\subsection}
{0pt}{2ex plus 0ex minus .2ex}{0ex plus .2ex}
\titlespacing*{\subsubsection}
{0pt}{2ex plus 0ex minus .2ex}{0ex plus .2ex}
\usepackage{sectsty}
\sectionfont{\centering}
\renewcommand{\thesection}{\Roman{section}.}
\renewcommand \thesubsection{\arabic{section}.\arabic{subsection}}

\begin{document}
\begin{center}
{\Large Программа дисциплины <<Анализ данных для лингвистов>>}
\end{center}
\begin{flushright}
\begin{tabular}{l}
Утверждена                          \\
Академическим советом ОП            \\
Протокол № 15 от  «28» июня 2018 г.
\end{tabular}
\end{flushright}
\begin{tabular}{|l|l|}
\hline
Автор                         & Г. А. Мороз                    \\ \hline
Число кредитов                & 3                              \\ \hline
Контактная работа (час.)      & 36                             \\ \hline
Самостоятельная работа (час.) & 78                             \\ \hline
Курс                          & 3, 4                            \\ \hline
Формат изучения дисциплины    & без использования онлайн-курса \\ \hline
\end{tabular}
\section{ЦЕЛЬ И РЕЗУЛЬТАТЫ ОСВОЕНИЯ ДИСЦИПЛИНЫ}
Задачей курса <<Анализ данных для лингвистов>> является продолжение знакомства с различными методами анализа данных. Курс разделен на несколько тематических блоков: первый связан с применением байесовских статистических методов (байесовский апдейт, байесовский доверительный интервал, байесовский фактор, байесовкская эмпирическая оценка), второй связан с методами уменьшения размерности (PCA, LDA, CA, MCA), третий блок связан с методами кластеризации (k-means, иерархическая кластеризация, смешанные модели) и последний блок будет посвящен проблемам применения регрессионного анализа (регрессия со смешанными эффектами, обобщённая аддитивная модель).
\section{СОДЕРЖАНИЕ УЧЕБНОЙ ДИСЦИПЛИНЫ}
\begin{enumerate}
\item работа с разными распределениями
\item байесовские статистические методы
\item кластеризация и смешанные модели
\item уменьшение размерности
\item проблемы применения регрессионного анализа
\end{enumerate}
\section{ОЦЕНИВАНИЕ}
Итоговая оценка складывается из следующих компонентов:
\begin{itemize}
\item текущий контроль домашнего чтения с весом 40\%,
\item экзамен с весом 60\%, в том числе дополнительная задача на оценку выше 8.
\end{itemize}
Текущий контроль складывается из оценок за контрольные работы и зачета в конце четвертого модуля. Результирующая оценка выставляется по следующей формуле:
$$O_\textnormal{итоговый} = 0.6 \times O_\textnormal{экзамен} + 0.4 \times O_\textnormal{текущий контроль}$$
Способ округления накопленной оценки промежуточного (итогового) контроля в форме зачета: арифметический.
\section{ПРИМЕРЫ ОЦЕНОЧНЫХ СРЕДСТВ}
\begin{itemize}
\item Посчитайте значение правдоподобия распределения $\mathcal{N}(\mu = 22,\, \sigma^{2}=6)$ для двух наблюдений 57 и 43.
\item Проведите байесовский апдейт данных $Beta(11, 34)$, используя априорное распределение $Beta(23, 45)$, и посчитайте симметричный 95\% байесовский доверильный интервал и 95\% интервал максимальной плотности.
\end{itemize}
\section{РЕСУРСЫ}
\subsection{Основная литература}
\subsection{Дополнительная литература}
\subsection{Программное обеспечение}
\subsection{Профессиональные базы данных, информационные справочные системы, интернет-ресурсы (электронные образовательные ресурсы)}
\subsection{Материально-техническое обеспечение дисциплины}
\end{document}
